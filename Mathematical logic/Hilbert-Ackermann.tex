% !TXS template
\documentclass[french]{article}

\usepackage[utf8]{inputenc}
\usepackage[bulgarian]{babel}
\usepackage{amsmath}
\usepackage{cancel}
% Title Page


\title{Теорема на Хилберт-Акерман}
\author{Изпит по "Математическа логика"}
\date{}


\begin{document}
\maketitle
\author

\subsection*{Дефиниция: Консервативно разширение.}
Нека $F$ е формална система. Нека $F'$ е разширение на $F$. Тоест: $F$ $\subseteq$ $F'$.
Ще казваме, че $F'$ е \textbf{консервативно} разширение на $F$, ако за всяка формула $A$ е изпълнено, че: ако $\vdash_{F'} A$, то  $\vdash_F A$.

\subsection*{Теорема за константите}
Нека $F'$ се получава от $F$ чрез добавяне на нови константи. Тогава $F'$ е консервативно разширение на $F$. 	
При това ако $x_1 \dots x_n$ са различни променливи и $c_1 \dots c_n$ са различни нови константи, то за всяка формула A:
\begin{align*}
\vdash_F  A  \iff \vdash_{F'} A_{x_1\dots x_n}[c_1\dots, c_n]
\end{align*}

\subsection*{Дефиниция: Противоречива формална система.}
Нека $F$ е формална система. $F$ е противоречива, ако за някоя формула A:
\begin{align*}
\vdash_F  A  \; \& \;   \vdash_F  \neg A
\end{align*}
Тогава следното твърдение е изпълнено: \newline
$F$ е противоречива $ \iff $ за всяка формула B е изпълнено, че $\vdash_F B$

\subsection*{Дефиниция: Хенкинова система.}
Една формална систма е \textbf{хенкинова}, ако за всяка затворена формула $\exists x A$ на системата, съществува затворен терм $t$, такъв че $\exists x A \implies A_x[t]$ е теорема на системата. \newline\newline
\textbf{Забележка.} А изразява свойство на x. Системата трябва да е \textit{готова с пример, в случай че свойството е изпълнено в нея}. Но това \textbf{НЕ} означава, че тя \textit{дава примера преди да сме доказали, че свойството е вярно.} 
\subsection*{Дефиниции: Специални константи. Специални аксиоми}
Нека $F$ е формална система. Ако $\exists x A$ е затворена формула от симовлите на $F$ (и специалните константи на $F$), то символът $K_{\exists x A}$ е \textbf{специална константа} на $F$ за формулата $\exists x A$. \newline
Ако $K_{\exists x A}$ е специална константа, то формулата: 
\begin{align*}
\exists x A \implies A_x[K_{\exists x A}]
\end{align*}
наричаме $\textbf{специална аксиома}$ за $K_{\exists x A}$.\newline\newline
Ако $F'$ е получен от $F$, добавяйки всички специални константи, то от $\textbf{теоремата за константите}$ следва, че $F'$ е консервативни разшиерение на $F$. А ако $SA$ е множеството на всички специални аксиоми, то $F'[SA]$ е \textbf{хенкинова формална система}.

\subsection*{Теорема 1:} $F'[SA]$ е консервативно хенкиново разширение на $F$.
\subsection*{Доказателство:}
Нека A е формула на $F$ и нека $\vdash_{F'[SA]} A$. От \textbf{теоремата за редукцията} имаме че (*):
\begin{align*}
\vdash_{F'} A_1 \implies A_2 \implies \dots \implies A_n \implies A
\end{align*}
където $A_i (1<= i <= n)$ са специални аксиоми. От \textbf{теоремата за тавтологиите}, може да считаме, че $A_1$ се отнася за специална константа от най-високо ниво. Тогава $A_1 \equiv \exists x B \implies B_x[C_ {\exists x B}]$. Да разгледаме $A_1$. Нека y е нова променлива. Тогава от \textbf{теоремата за константите}:
\begin{align*}
\vdash_{F'} \exists x B \implies B_x[y] \implies A_2 \implies \dots \implies A_n \implies A
\end{align*}
От (П $\exists$) имаме: 
\begin{align*}
\vdash_{F'} \exists y(\exists x B \implies B_x[y]) \implies A_2 \implies \dots \implies A_n \implies A
\end{align*}
\textbf{Забележка:} y не участва в $A_2 \dots A_n$ и $A$. Освен това избрахме $A_1$ да бъде от най-високо ниво. Следователно пак ще получихме тавтология. \newline
Съгласно пренексните опрации (и y не участва в $B$) имаме, че:
\begin{align*}
\vdash_{F'} \exists y(\exists x B \implies B_x[y]) \iff \vdash_{F'} (\exists x B \implies \exists y B_x[y]) 
\end{align*}
Но $ \exists y B_x[y]$ е вариант на $\exists x B$. Следователно: $\vdash_{F'} A_1$ От тук по \textbf{modus ponens}:
\begin{align*}
\vdash_{F'} \cancel{A_1} \implies A_2 \implies \dots \implies A_n \implies A
\end{align*}
Прилагайки тези разсъждения n-1 пъти получаваме, че $\vdash_{F'} A$, от където следва, че $\vdash_{F} A$

\subsection*{Твърдение 1:} Ако $\vdash_F$ $T$, то всеки затворен частен случай на $T$ от $L_c(F)$ е тавтологично следствие на затворени частни случаи в $L_c(F)$ на аксиомите и специалните аксиоми на $F$. 
\subsection*{Доказателство:} 
Нека $\vdash_F$ $T$ и нека $T'$ е затворен частен случай на $T$. \newline Ще покажем, че $T'$ е тавтологично следствие на затворени частни случаи на аксиомите и специалните аксиоми. Ще направим индукция по извода в $F$.
\newline\newline
\textbf{1.} Ако $T$ не се получва чрез (П $\exists$).

\begin{itemize}
\item Ако $T$ е аксиома,
\end{itemize}
 то $T'$ е затворен частен случай на аксиома на $F$. \newline \textbf{Но всеки терм е тавтологично следствие на себе си} (в частност $T'$ на $T'$), от където следва, че $T'$ е тавтологично следствие на затворени частни случай на аксиомите на $F$ (в частност аксиомата $T$).
 \begin{itemize}
\item Ако $T$ е тавтологично следствие на $T_1, T_2 ... T_n$:
\end{itemize}
Тогава $T'$ се получава чрез замяна на свободните променливи с термове в $T$. Прилагаме същите замени в $T_1, T_2 ... T_n$ и получаваме  $T'_1, T'_2 ... T'_n$. Тогава $T'_1, T'_2 ... T'_n$ са затворени частни случаи на теоремите $T_1, T_2 ... T_n$. \newline \textbf{Oт и.п.} $T'_1, T'_2 ... T'_n$ са тавтологични следствия на затворени частни случаи на аксиомите и специалните аксиоми. Но $T'$ e тавтологично следствие на $T'_1, T'_2 ... T'_n$. A  $T'$ е тавтологично следствие на това, на което $T'_1, T'_2 ... T'_n$ са тавтологични следствия. \newline Следователно $T'$ е тавтологично следствие на затворени частни случаи на аксиомите и специалните аксиоми. 
\newline\newline
	\textbf{2.} Ако $T$ се получава от $A \implies B$ чрез (П $\exists$).
\newline 
	Тогава: $T' = \exists x A' \implies B'$, където $A'$ е частен случай на $A$, а $B'$ е \textbf{затворен} частен случай на B.\newline
	Възможно е А' да \textbf{не е затворена}. Ако в $A$ променливата $x$ е участвала \textbf{свободно}, то в $T'$ тя вече е свързана и \textbf{не можем да я заменим} в $A'$. \newline
	Тогава за да получим \textbf{затворен} частен случай на  $A \implies B$ \textbf{трябва да заменим свободните срещания на } $x$ (които са в $A'$) с терм. \newline
	Нека $C_{\exists x A'}$ е \textbf{специалната константа} на $\exists x A'$. \newline
	Тогава $A'_{x}[C_{\exists x A'}] \implies B'$ \textbf{е затворен частен случаи на} $A \implies B$.
	От \textbf{и.п.} формулата $A'_{x}[C_{\exists x A'}] \implies B'$ е тавтологично следствие на аксиомите на $F$ и специалните аксиоми на $F$. \newline
	Специалната аксиома за $C_{\exists x A'}$ е $\exists x A' \implies 
	A'_{x}[C_{\exists x A'}] $.
	Имаме, че: 
	\begin{itemize}
		\item[a)]  $\exists x A' \implies A'_{x}[C_{\exists x A'}]$ е специална аксиома.
		\item[б)] $A'_{x}[C_{\exists x A'}] \implies B'$  е тавтологично следствие на аксиомите и специалните аксиоми на $F$
		
	\end{itemize}
\textbf{От а), б) и транзитивността на импликацията} следва, че: \newline
 $T' = \exists x A' \implies B'$ е тавтологично следствие на затворени частни случаи в $L_c(F)$ на аксиомите и специалните аксиоми на $F$.

\subsection*{Теорема на Хилберт-Акерман:}
Нека $F$ е формална система, имаща само \textbf{безкванторни нелогически аксиоми}. Тогава $F$ е противоречива тогава и само тогава, когато някоя дизюнкция от отрицания на частни случаи на нелогическите аксиоми на $F$ е тавтологично следствие на частни случаи на аксиомите за равенството.
\subsection*{Доказателство:}
$\impliedby$ ) Нека някоя дизюнкция от отрицания на частни случаи на нелогически аксиоми на $F$ е тавтологично следствие на частни случаи на аксиомите за равенството.Т.е. : 
\begin{align*}
\neg A_1 \lor \neg A_2 \lor \dots \lor \neg A_n
\end{align*}
е тавтологично следствие на частни случаи на аксиомите за равенството и $A_i (1 <= i <= n)$ е частен случай на аксиомите на $F$.
Но тогава:
\begin{align*}
 \vdash_F \neg A_1 \lor \neg A_2 \lor \dots \lor \neg A_n
\end{align*}
но $A_i$ е частен случай на аксиома и съгласно \textbf{теоремата за замяната}: 
\begin{align*}
\vdash_F  A_1, \vdash_F A_2,  \dots \vdash_F  A_n
\end{align*}
Горната формула може да запишем така:
\begin{align*}
 A_1 \implies A_2 \implies A_3 \implies \dots \implies A_{n-2} \implies A_{n-1} \implies \neg A_n.
\end{align*}
Но по \textbf{modus ponens}:  
\begin{align*}
 \cancel{A_1} \implies A_2 \implies A_3 \implies \dots \implies A_{n-2} \implies A_{n-1} \implies \neg A_n 
\end{align*} 
\begin{align*}
  \cancel{A_1} \implies \cancel{A_2} \implies A_3 \implies \dots \implies A_{n-2} \implies A_{n-1} \implies \neg A_n 
\end{align*}
\begin{align*}
\cancel{A_1} \implies \cancel{A_2} \implies \cancel{A_3} \implies \dots \implies A_{n-2} \implies A_{n-1} \implies \neg A_n 
\end{align*}
\begin{align*}
\vdots
\end{align*}
\begin{align*}
\cancel{A_1} \implies \cancel{A_2} \implies \cancel{A_3} \implies \dots \implies \cancel{A_{n-2}} \implies \cancel{A_{n-1}} \implies \neg A_n 
\end{align*}
Следователно $\neg A_n$ \textbf{е теорема} на $F$. Но и $A_n$ е \textbf{теорема} на $F$. \newline От тук следва, че $F$ е \textbf{противоречива}. \newline \newline
$\implies )$ Нека $F$ е противоречива (и $F$  има само безкванторни нелогически аксиоми). Тогава $\vdash_F x \neq x $. Но от твърдение 1, следва, че $c \neq c$ ($c$ е произволна константа от $L_c(F)$) е \textbf{тавтологично следствие} на затворени в $L_c(F)$ частни случай на аксиомите и специалните аксиоми на $F$. \newline
Т.е от затворени частни случай на аксиомите и специалните аксиоми на $F$, използвайки само \textbf{теорема за тавтологиите}, може да докажем, че $c \neq c$.
Следователно имаме нещо от типа:
\begin{align*}
A_1 \implies A_2 \implies A_3 \implies \dots \implies A_n \implies c \neq c
\end{align*}
което е тавтология. $A_i (1<= i <= n)$ е частен случай на аксиомите и специалните аксиоми на $F$.
Ще трябва да се освободим от \textbf{аксиомите за субституция и специалните аксиоми}.
Така ще получим тавтология, която е дизюнкция от отрицания на затворени частни случаи в $L_c(F)$ на аксиоми за равенството и нелогически аксиоми на $F$.
Имаме следната тавтология:
\begin{align*}
\neg A_1 \lor \neg A_2 \lor \neg A_3 \dots \lor \neg A_n
\end{align*}
$A_i (1<= i <= n)$
\begin{itemize} 
	\item Затворена аксиома за субституцията.
	\item Специална аксиома.
	\item Затворен частен случай на аксиома за равенството.
	\item Затворен частен случай на нелогическа аксиома на $F$.
\end{itemize}
Но $F$ е отворена формална система - \textbf{нелогическите аксиоми нямат квантори}. В аксиомите за равенството също \textbf{няма квантори}.
Тогава може да разпознаем дали $A_i$ е аксиома за субституцията/специална аксиома по това \textbf{дали съдържа квантор}. \newline
Нека $\exists x A'$ е формулата с възможно най-много квантори от тавтологията.
Тогава $\exists x A'$ е парче от: 
\begin{itemize} 
	\item $A'_x[a] \implies \exists x A'$ (част от аксиома за субституцията).
	\item $\exists x A' \implies A'_x[C_{\exists x A'}]$ (част от специална аксиома)
\end{itemize}
Тогава имаме следната тавология:
\begin{align*}
\neg (\exists x A' \implies A'_x[C_{\exists x A'}]) \lor 	\neg (A'_x[a_1] \implies \exists x A') \lor \dots 	\lor \neg(A'_x[a_m] \implies \exists x A')  \lor 
\end{align*}
\begin{align*}
\lor \neg A_1 \lor \neg A_2 \lor \neg A_3 \dots \lor \neg A_s
\end{align*}
където $\exists x A'$ не участва в $A_1, \dots, A_s$. 
\newline
\textbf{Забележка:} Ако някой дизюнкт не участва във формулата \newline (например: $A'_x[a] \implies \exists x A'$ ), то понеже нашата дизюнкция е тавтология, можем да го добавим. \newline
Искаме да премахнем $\exists x A'$. След краен брой прилагания на тази процедура(която сега ще приложим), ще премахнем всички квантори от тавтологията. \newline \newline
\textbf{Заместваме} $\exists x A'$ с  $A'_x[C_{\exists x A'}])$
\begin{align*}
\neg ( A'_x[C_{\exists x A'}] \implies A'_x[C_{\exists x A'}]) \lor 	\neg (A'_x[a_1] \implies A'_x[C_{\exists x A'}]) \lor \dots 	\lor \neg(A'_x[a_m] \implies A'_x[C_{\exists x A'}])  \lor 
\end{align*}
\begin{align*}
\lor \neg A_1 \lor \neg A_2 \lor \neg A_3 \dots \lor \neg A_s
\end{align*}
Но $A'_x[C_{\exists x A'}] \implies A'_x[C_{\exists x A'}]$ е аксиома, т.е целият дизюнкт можем да го премахнем. Така получаваме тавтологията $(*)$:
\begin{align*}\neg (A'_x[a_1] \implies A'_x[C_{\exists x A'}]) \lor \dots 	\lor \neg(A'_x[a_m] \implies A'_x[C_{\exists x A'}])  \lor 
\end{align*}
\begin{align*}
\lor \neg A_1 \lor \neg A_2 \lor \neg A_3 \dots \lor \neg A_s
\end{align*}
За всеки израз $t$ с $t^i$ да означим израза, който се получава след заместване на $C_{\exists x A'}$ с $a_i$. \textbf{Заместваме и в индексите на специалните константи}. \newline
При тази замяна, \textbf{спциалната константа се превръща в специална константа}, а \textbf{специалните аксиоми в специални аксиоми}. \newline
Получаваме m тавтологии.
\begin{align*}
\neg (A'_x[a_1^1] \implies A'_x[a_1]) \lor \dots 	\lor \neg(A'_x[a_m^1] \implies A'_x[a_1])  \lor \neg A_1^1 \lor  \dots \lor  \neg A_s^1
\end{align*}
\begin{align*}
\neg (A'_x[a_1^2] \implies A'_x[a_2]) \lor \dots 	\lor \neg(A'_x[a_m^2] \implies A'_x[a_2])  \lor \neg A_1^2 \lor \dots \lor \neg A_s^2
\end{align*}
\begin{align*}
\vdots
\end{align*}
\begin{align*}
\neg (A'_x[a_1^m] \implies A'_x[a_m]) \lor \dots 	\lor \neg(A'_x[a_m^m] \implies A'_x[a_m])  \lor \neg A_1^m \lor  \dots \lor  \neg A_s^m
\end{align*}
Ако в i-тата тавтология $A'_x[a_i]$ е истина, то първите m дизюнкта ще са лъжа. От тук получаваме, че:

\begin{align*}
 A'_x[a_1] \implies \neg A_1^1 \lor  \dots \lor  \neg A_s^1
\end{align*}
\begin{align*}
A'_x[a_2] \implies \neg A_1^2 \lor  \dots \lor  \neg A_s^2
\end{align*}
\begin{align*}
\vdots
\end{align*}
\begin{align*}
A'_x[a_m] \implies \neg A_1^m \lor  \dots \lor  \neg A_s^m
\end{align*}
От $(*)$  за да бъде истина дизюнкта $\neg (A'_x[a_1] \implies A'_x[C_{\exists x A'}]) (1<=i<=m)$, то трябва $A'_x[a_1]$ да е истина. Следователно получаваме:

\begin{align*}
\neg (A'_x[a_1] \implies A'_x[C_{\exists x A'}]) \implies A'_x[a_1]
\end{align*}
\begin{align*}
\neg (A'_x[a_2] \implies A'_x[C_{\exists x A'}]) \implies A'_x[a_2]
\end{align*}
\begin{align*}
\vdots
\end{align*}
\begin{align*}
\neg (A'_x[a_m] \implies A'_x[C_{\exists x A'}]) \implies A'_x[a_m]
\end{align*}
Следователно следната формула е тавтология:
\begin{align*}
\neg A_1^1 \lor \dots \lor \neg A_s^1 \lor \dots \dots.  \lor \neg A_1^m \lor \dots \lor \neg A_s^m \lor \neg A_1 \lor \dots \neg A_m.
\end{align*}
Нека Б.О.О в горната формула няма квантори (ако има, то ще приложим отново горната стратегия). Всеки дизюнкт е частен случай или на аксиомите за равенството, или на нелогическите аксиоми за $F$. Нека с $\alpha_i$ бележим i-тият дизюнкт, който е частен случай на аксиома за равеснтвото, а с $\beta_i$ i-тият дизюнкт, който е частен случай на нелогическите аксиоми на $F$. (броим от ляво надясно). 
Тогава следното е тавтология (след преподреждане):
\begin{align*}
\alpha_1 \implies \alpha_2 \implies \dots \implies \alpha_k  \implies (\neg \beta_1 \lor \dots \lor \neg \beta_r) 
\end{align*}
Получихме, че дизюнкцията от отрицания на частни случай на нелогическите аксиоми на $F$ $(\neg \beta_1 \lor \dots \lor \neg \beta_r)$ е тавтологично следствие на частни случаи на аксиомите за равенството ($\alpha_1 \dots \alpha_k$).
\end{document}
